\documentclass[12pt,a4paper]{article}
\title{Experiment 14: Linux Kernel}
\author{Aarya R Shankar} 
\date{May 13, 2017}
\begin{document}
\maketitle
\begin{center}
To view the source code:

\texttt{\ https://github.com/torvalds/linux/tree/master/Documentation}
\end{center}
\section{Aim}
 
Kernel configuration, compilation and installation : Download/access the latest kernel source code from kernel.org, compile the kernel and install it in the local system. Try to view the source code of the kernel.

\section{Procedure}

First download the latest kernel from kernel.org using curl or wget.\\
\newline
\texttt{\ curl https://cdn.kernel.org/pub/linux/kernel/v4.x/linux-4.11.tar.xz -o linux-4.11.tar.gz}\\
\newline
The downloaded file is in tar.gz format. To unzip it, do\\
\newline
\texttt{\ tar -xvf linux-4.11.tar.gz}\\
\newline
You will see a list of files that is being unzipped from the downloaded file.
Now if you do ls, you will see a new directory named linux-4.11
Move into that directory\\
\newline
\texttt{\ cd linux-4.11}\\
\newline
Now that we have downloaded our kernel source code and expanded it, we need to make a configuration file, .config, to configure the kernel with our desired options.
Note: We can create the configuration file from scratch (with our desired options), or based on a default configuration, or based on the currently running kernel verion, or based on some kernel release distribution. Here first, we'll create a .config file from scratch\\

\texttt{\ make config}\\
\newline
After running this command, you'll be required to answer a number of questions like:\\
\newline
\begin{enumerate}\item for which system, are you building the kernel (32 or 64 bit)
\newline\item if you want cross compile tools (press enter if you do not compile code to be run on other processors, else give "x86\texttt{\_}64-pc-linux-gnu-" for 64 bit PCs, or "arm-unknown-linux-gnu-" for ARM processor systems)
\newline\item what should be appended to our custom kernel version (ex: If you give firstcustombuild, it will display XXX-4.11-firstcustombuild)
\newline\item The kernel compression mode we prefer (Gzip is the default option)
\newline\item Our preferred default hostname (usually developers leave this blank, so that all linux users can chose their own hostname)
\newline\item Support for swap memory (If you have a dual boot, or are planning to get one, choose yes)
\newline\item Preference for IPC (Inter Process Communication) (It is always best to enable this, if you want all applications to run)
\newline\item Preference for POSIX Message Queues so that messages can be labelled with a priority (Choose yes !important)
\newline\item Preference for Cross Memory Attach so that privileged processes can write or read from other prcoesses' memory (Allowing it is preferred)
\newline
\item Preference for uselib syscalls (CONFIG\texttt{\_}USELIB) (If you're sure your system runs on glibc library, type no. Else run the command 'ldd --version' in another terminal, if you see your current glibc version, type n, else type yes)
\newline
\item Preference for IRQ Debug (type y if you want to know the mapping relationship between hardware irq numbers and Linux irq numbers. The mapping is exposed via debugfs in the file "irq\texttt{\_}domain\texttt{\_}mapping". Type n if you don't understand this)
\newline
\item Preference for Timer Tick handle (Choose NO\texttt{\_}HZ\texttt{\_}IDLE or NO\texttt{\_}HZ\texttt{\_}FULL for lesser cpu usage)
\newline
\item Preference for Old Idle dynticks config (If your system hardware is slow or old, choose no. Else type yes so that the timer interrupts can be used whenever needed, so as to enable tasks to be executed at particular interval of time)
\newline
\item Preference for High Resolution Timer (Most relatively modern systems (Pentium III and higher) have high resolution timers, allowing for more precise timing. Not really mandatory, but some applications like mplayer can benefit from using hi-res timers. You can manually check if your system supports high resolution timer by looking at the values given in /proc/timer\texttt{\_}list)
\newline
\item Preference for Cputime accounting (TICK\texttt{\_}CPU\texttt{\_}ACCOUNTING is preferred as it has simple accounting)
\newline
\item Fine granularity task level IRQ time accounting (IRQ\texttt{\_}TIME\texttt{\_}ACCOUNTING) (Default is no)
\newline
\item BSD\texttt{\_}PROCESS\texttt{\_}ACCT (It basically logs all the processes that runs in the kernel. Choose no if you want a smaller and faster kernel)
\newline
\item CPU/Task time and stats accounting (Some questions are already given as yes. For TASK\texttt{\_}XACCT, choose no if you want a small kernel as it collects extended task accounting and sends it to userland for processing over the taskstats interface).
\newline
\item CONFIG\texttt{\_}RCU\texttt{\_}EXPERT (Choose yes only if you want to make expert level modifications to Read-Copy-Update configurations)
\newline
\item IKCONFIG (This option makes the whole .config file to be save in the kernel with proper documentation, so that it can be extracted form the kernel image using the command scripts/extract-ikconfig and be used to rebuild the current kernel or to build another one. We can even extract it form a running kernel by reading /proc/config.gz)
\end{enumerate}\\
\newline
There are many more questions. There will be a '?' option along with y or n or m, we can choose that to know more about the question.\\
\newline
Now we want to make the configuration file based on the default configuration options (preferred, unless you really know what you're doing and have lots of time (hours) to read through all the questions and give the necessary answers. However avoiding the unnecessary options will make for a smaller and faster kernel).\\
\newline
\texttt{\ make defconfig}\\
\newline
For further manual configuration, after setting up the .config file, run\\
\newline
\texttt{\ make menuconfig}\\
\newline
or\\
\newline
\texttt{\ make xconfig screen}\\
\newline
If you want specific menu, give it as an attribute at the end of the command.\\
\newline
Now that we have a .config file, we can build the kernel by running\\
\newline
\texttt{\ make}\\
\newline
Now we can install it by running the installation script of the distribution, which may be used by the kernel build system to automatically install a built kernel into the proper location and modify the bootloader so that nothing extra needs to be done by the developer.\\
\newline
If you have built any modules and want to use use this method to install a kernel, first enter:\\
\newline
\texttt{\ make modules\texttt{\_}install}\\
\newline
This will install all the modules that you have built and place them in the proper location in the filesystem for the new kernel to properly find. Modules are placed in the /lib/modules/ KERNEL\texttt{\_}VERSION directory, where KERNEL\texttt{\_}VERSION is the kernel version of the new kernel you have just built.\\
\newline
After the modules have been successfully installed, we now need to install the main kernel image.\\
\newline
\texttt{\ make install}\\
\newline
This command will verify that the kernel has been successfully built properly, the build system will install the static kernel portion into the /boot directory and name this executable file based on the kernel version of the built kernel, etc\\
\newline
After this is finished, the kernel is successfully installed, and you can safely reboot and try out your new kernel image. Note that this installation does not overwrite any older kernel images, so if there is a problem with your new kernel image, the old kernel can be selected at boot time. 

\section{Result}

The linux kernel is successfully downloaded, modified, compiled, and installed as given.
\end{document}
